\documentclass{Tools}

\usepackage[french]{babel}

\begin{document}

\title[]{.}

\author{N. VERTUEUX}

\date{February 20, 2023}

\dedicatory{}

\begin{abstract}
\end{abstract}

\maketitle

\textbf{Question 0.1.} \emph{Qu'est-ce que le temps? Est-ce une notion factuelle ou fondée?}\\ 

\begin{proof}
Le temps, outils fondamental et indispensable de la physique, nous a permis de décrire le monde autour de nous. Nous le considérons comme une notion
simple et admise par tous. Qu'il s'agit d'une notion linéaire ou relative, le temps est fondamental à la vie de tout être vivant.

Ma première pensée supposerait que le temps est une notion inventée par l'homme et la femme. Nous l'avons inventé pour pouvoir nous repérer et bâtir le 
monde que nous connaissons aujourd'hui.

Cependant, lorsque nous souhaitons pousser l'idée à son maximum, nous pouvons nous rendre compte que malgré que le temps soit une idée considérée à l'origine par l'homme, cela ne répondra pas à la question de savoir s'il a déjà existé auparavant. Mais ce problème présenté s'applique malheureusement à toutes les notions de la vie, et plus particulièrement la métaphysique. 

Nous voyons le monde sous notre perception individuelle, il est notamment impossible de démontrer que toutes les interactions que vous avez dans le monde sont réelles, et ne sont pas fondées par vous-même, et cela relèvera la question de savoir si nous sommes un ensemble ou une singularité. Nous voyons le monde à travers nos différents sens, mais s'il on était dans la capacité de tous les retirer, que ce serait-ce la vie ?

Une première approche serait de dire que nous serons concrètement une simple pensée, déconnectée de la réalité, du monde autour de lui. Comment décriveriez-vous un instant précédent face à un instant suivant. Qu'est-ce qui ne vous dirait pas que dans trois seconde pour votre cerveau se serons écoulés 50 ans pour une personne externe et que vous allez mourir dans 1 seconde? 

Il s'agit d'un vide de pensée, soutenu par l'absence d'une connection externe. Le temps n'est qu'une unité de mesure cachée au fond du subconscient de chaque homme.
Sans conscience, le temps demeure inexistant.

Sous la vision d'un temps existant en dehors de notre conscience, rien ne serait capable de nous prouver que ce temps avance, est en arrêt ou même qu'il recule.
Qu'est-ce qui vous permettrait de dire qu'il n'y a pas eu une pause infinie entre vous lisant le mot "infinie" et la fin de cette phrase? 

En réalité, cela décrirait un problème encore plus complexe. Comment pourrions nous décrire le temps passé lorsque que notre temps a été mis en pause? Afin de pouvoir déterminer ce qu'est le temps, nous devrions être capable de déterminer ce qu'il a été. Et qu'est-ce que serait le présent sans passé ? Sans passé, le présent demeure un instant inateignable . Et sans présent, le futur n'est qu'un moment non-existant.

Malgré que cette idée pourrait sembler étrange, et qu'il serait impossible de mesurer le temps qui sera arrêté, cela démontre clairement que le temps est une notion impossible à décrire sans savoir ce qu'il est déjà.

Malgré que le passé est inscrit dans le temps, et que le futur sera inscrit dans l'histoire, le présent est une notion inexistante, à l'heure où vous lisez ce texte, ce moment sera considéré dans le passé. Le présent est tel une limite impossible à atteindre.

Nous conceptualisons le temps sous trois axes: le passé, le présent et le futur. Mais si je vous disais que cela n'était pas le cas, et que le temps pourrait être représenté sous 4, 5 ou même une infinité d'axe, comment pourriez-vous le voir? Notre cerveau et notre perception est fondée sur une notion du temps apprise

\end{proof} 

\textbf{Question 0.2.} \emph{Comment est-ce que le temps nous affecte ?}\\ 
Le temps est un concept considéré indispensable dans la vie de tous les jours. Mais l'est-il réellement ?

\bibliographystyle{amsplain}

\end{document}