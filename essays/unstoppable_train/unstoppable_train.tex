\documentclass[twocolumn, french]{article} 

\usepackage[margin=2cm]{geometry}  
\usepackage{titling}
\usepackage[main=french]{babel}             

\setlength{\droptitle}{-2cm}    
\author{\large{\textbf{N. VERTUEUX}}}
\title{
  \fontsize{30pt}{36pt}\selectfont \textbf{Identitas}
}

\begin{document}
\maketitle
\section*{Un Train inarrêtable} 
Inventons un conte. Imaginons un homme. Il s'agit d'un homme rempli d'ambition, d'orgueil, rigoureux et 
souhaitant faire changer les choses. Il travaillait et vivait dans une gare déprimante. 

Ces journées furent longues et ennuyeuses, surtout quand il imaginait que l'administration de la gare le 
manipulait et profitait de lui, pour leur propre plaisir. 

Cependant, un jour, il vit un train neuf et propre, avec un panneau sur la portière, avec marqué de ne surtout pas 
monter dessus et de démarrer le trajet, car cela serait susceptible de mener la gare à sa perte, tel une pomme 
proposée par un serpent. Mais d'après certains, ce train les mèneraient vers une meilleure gare, sauf que 
l'administration le leur cache, car quitter cette gare leur donnerait l'opportunité de pouvoir découvrir de nouvelles 
choses et d'évoluer, les mettant à un pied égal voire supérieur que ces derniers.   

L'homme, malmené et soumis à sa maléfique curiosité, décida de ne pas respecter le panneau 
et embarqua des passagers avec lui, en leur promettant une gare meilleur, en espérant quitter 
cette prison dépressive, pour en trouver une qui pourrait être plus appréciable. Cependant, après 
avoir démarré le train, ce pauvre homme malencontreusement détruisit le frein à main, et le train fragile, 
commença à avancer avec une vitesse décuplant toutes les secondes, et ce dernier devint inarrêtable. 

Les passagers, à l'arrière du train, n'eurent aucune conscience de ce qu'il se passait à l'avant, car,
ce train était en réalité rempli de tous les divertissements et activités possibles afin de faire passer le temps, 
comme si les casinos sont désormais des trains.
Sauf que certains commençanient à prendre conscience que quelque chose n'allait pas, cependant, personne ne 
souhaitait les écouter, ils les prenaient pour des fous pessimistes, et de plus, la porte menant à l'avant du train 
était fermée, avec le conducteur. Certaines personnes conscientes auront essayés de sauter et sortir du train, sauf 
que cela causera bien évidemment, leurs perte.

La voie ferrée aura bien une fin, cependant, pas avec la gare auxquels ils s'y attendaient.

Honnêtement je suis certain qu'ils créeront un nouveau frein à main afin d'arrêter le train, mais ce sera bien 
sûr, quand ils verront le bout du tunnel. \\
Après qui dira si ils y arriveront, moi je pense que oui, peut-être avec l'aide de l'administration, qui sait, mais 
ça ne passera pas loin...
\section*{Introduction}
Maintenant, disons que la gare et le tunnel représente notre monde, le train représente la société, les 
passagers représentent les citoyens, et l'administration représente ce qui nous dépasse, le divin.

L'avant du train avec le conducteur représente le gouvernement. La vitesse exponentielle du train représente le fait 
que l'on s'enfonce de plus en plus dans le trou que l'on aura creusé, étant la société. Le chemin empreinté par le 
train représente les actes réalisés par la société.

Et enfin, les personnes essayant de sauter du train représentent le fait que nous sommes dépendants de notre société, 
sans elle nous ne pouvons plus rien faire, mais elle a malheureusement une destination imprévisible et personne n'est 
en mesure de l'arrêter, pour l'instant.
\end{document}