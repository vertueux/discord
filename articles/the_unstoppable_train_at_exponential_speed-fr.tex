\documentclass[twocolumn, french]{article} 

\usepackage[margin=2cm]{geometry}  
\usepackage{titling}
\usepackage[main=french]{babel}             

\setlength{\droptitle}{-2cm}    
\author{\large{\textbf{N. VERTUEUX}}}
\title{
  \fontsize{30pt}{36pt}\selectfont \textbf{Le train inarrêtable à grande vitesse exponentielle}
}

\begin{document}
\maketitle
\section*{Introduction} 
Inventons un conte. Imaginons un homme. Il s'agit d'un homme rempli d'ambition, d'orgueil, rigoureux et 
souhaitant faire changer les choses. Il travaillait dans une gare déprimante. 
\\ 
Ces journées furent longues et ennuyeuses, surtout quand il imaginait que l'administration de la gare le 
manipulait et profitait de lui, pour leur propre plaisir. Cependant, un jour, il vit un train neuf et propre, 
avec un panneau sur la portière, avec marqué de ne surtout pas monter dessus et de démarrer le trajet, car cela 
serait susceptible de mener la gare à sa perte. Mais d'après certains, ce train les mèneraient vers une 
meilleure gare, mais l'administration de la gare le leur cache, car quitter cette gare leurs donneraient 
l'opportunité de pouvoir découvrir de nouvelles choses et d'évoluer, les mettant à un pied égal voire supérieur 
que ces derniers.   
\\
L'homme, malmené et soumis à sa maléfique curiosité, décida de ne pas respecter le panneau 
et embarqua des passagers avec lui, en leur promettant un monde meilleur, en espérant quitter 
cette gare dépressive, pour en trouver une qui pourrait être plus appréciable. Cependant, après 
avoir démarré le train, ce pauvre homme malencontreusement détruisit le frein à main, et le train fragile, 
commença à avancer avec une vitesse décuplant toutes les secondes, et ce dernier devint inarrêtable. 
\\
Les passagers, à l'arrière du train, n'eut aucune conscience de ce qu'il se passait à l'avant, car,
ce train était en réalité rempli de tous les divertissements possibles afin de faire passer le temps, 
comme si les casinos sont désormais des trains.
Sauf que certains commençanient à prendre conscience que quelque chose n'allait pas, cependant, personne ne 
souhaitaient les écouter et la porte menant à l'avant du train était fermée, avec le conducteur.
Certaines personnes conscientes auront essayés de sauter et sortir du train, sauf que cela causera leurs perte.
\\
La voie ferrée aura bien une fin, cependant, pas avec la gare auxquels ils s'attendaient.
\\ \\
Maintenant, disons que la gare représente notre monde, le train représente la société, les 
passagers représentent les citoyens, l'avant du train avec le conducteur représente l'homme qui, 
a l'essence même eu l'idée d'inventer la société, et l'administration représente ce qui nous dépasse, le divin.
La porte fermée menant au conducteur représente le fait que l'homme ayant fait naître la 
société est désormais mort, maintenant, au XXIe siècle. La vitesse exponentielle du train représente le fait que 
l'on s'enfonce de plus en plus dans le trou que l'on aura creusé, étant la société. 
\\
Et enfin, les personnes essayant de sauter du train représentent le fait que nous sommes dépendants de notre société, 
sans elle nous ne pouvons plus rien faire, mais elle a malheureusement une destination imprévisible et personne n'est 
en mesure de l'arrêter, pour l'instant.
\end{document}