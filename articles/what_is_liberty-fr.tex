\documentclass[twocolumn, french]{article} 

\usepackage[margin=2cm]{geometry}  
\usepackage{titling}     
\usepackage[main=french]{babel}        

\setlength{\droptitle}{-2cm}    
\author{\large{\textbf{N. VERTUEUX}}}
\title{
  \fontsize{30pt}{36pt}\selectfont \textbf{Qu'est-ce que la liberté ?}
}

\begin{document}
\maketitle
\section*{Introduction}

La liberté dans notre société actuelle est une source à débat parmis tous les individus
et politiciens de ce monde, en argumentant du droit de personnes fortunés, des personnes
détenant des difficultés financières, l'inégalité envers les autres et surtout une 
comparaison globale, car malgré que ces deux notions sont complètement différentes, elles
sont intimement liés. Cependant, la question et l'idéologie que soulèvera ce papier est 
sur la notion de cette dernière, en effet, remettre en question ce qu'est concrètement la 
liberté, pourquoi et comment la société nous empêchent indirectement d'y accéder, comment 
cette dernière lui attribue une fausse définition et l'utilise comme devise nationale, 
notamment en France, pourquoi n'existe-t-elle pas à proprement parlé et quelle est elle 
réellement. 
\section*{Argumentation}
\subsection*{I - La société}
Si l'on se base sur une définition universelle de ce qu'est la liberté, nous pouvons la 
résumer à l'absence de contrainte, la non-soumission à la servitude et la capacité de la 
conscience d'agir et de penser selon la volonté de l'individu. \\

Cependant, la société nous offre une liberté mensongère implicite. En effet, cette liberté 
offerte par le gouvernement est semblable à autoriser un animal de se balader dans sa cage, 
une cage dorée, détenant tous les outils possibles pour le divertir et le faire oublier quel 
est son réel droit, comme le dit le gouvernement, qu'est la liberté. 
Nous, individus sommes donc noyés et manipulés dans tous les divertissements, modes, 
formalités et conformisme imposés par notre société.

Notre société actuelle en France divise libertés et droits fondamentaux en quatre parties, 
je cite: "Les droits inhérents à la personne humaine", "les droits qui sont des aspects ou 
des conséquences des précédents", "les droits sociaux et économiques", puis "les droits dits 
"de troisième génération". Tout cela qui implique déjà une modification et redéfinition de la 
liberté. Qu'est-ce qu'une société démocratique qui se donne des valeurs et objectifs à 
atteindre, en modifiant leur sens ? Dans ce cas là, nous pourrions donner n'importe quel 
fausse valeur que nous respecterions facilement.
\subsection*{II - L'influence}
Dans le monde d'aujourd'hui, au XXIe siècle, nous oublions et perdons notre notion de la 
liberté (et principalement la liberté de nos choix) face à tous les objets et personnes de notre 
quotidien. Cela commence, et est en majeur partie face aux réseaux sociaux, notre téléphone 
intelligent portable et les personnes de notre entourage. \\
Nous ne sommes plus et n'avons jamais été indépendants, nous dépendons de notre environnement.
Les réseaux sociaux primaires étant installés sur nos écrans tissent un lien entre notre 
subconscient et ces logiciels, de part le fait d'avoir l'excitement de recevoir une 
notification de la part d'un proche, de voir notre le nombre d'abonnés augmenter, de passer de 
nombreuses heures à trouver une publication qui pourra nous divertir et nous faire exprimer
des émotions.  \\
En réalité, ces logiciels nous font oublier le reste de notre environnement et entourage car il 
aspire et intrigue notre subconscient, qui s'efforcera de contempler et chercher, par curiosité.
\\ Le fait de vouloir constamment regarder et essayer d'augmenter notre nombre d'abonnés est lié au
fait que l'on a tous ce rêve de pouvoir devenir populaire parmis le reste de la population, car cela 
est même devenu un critère pour la majorité des individus de notre société afin d'être apprécié et être 
en couple, spécialement chez la jeunesse, car le concerné va sentir que sa potentielle âme soeur
est appréciée, "validée" par beaucoup d'autres individus, et qu'il ne s'agit pas de n'importe qui.
\\ L'excitement du fait de recevoir une notification de la part d'un proche est tout simplement lié 
au fait que l'on aime avoir ce concept qu'est le fait d'être apprécié et pris en compte, mais nous 
avons tendance à oublier que nous pouvons le faire dans la vie réelle, car le faire sur les 
réseaux sociaux est bien plus facile et nous évite à faire beaucoup d'effort et sortir de notre 
zone de confort. \\ Dernièrement, nous passons des heures à chercher et regarder des publications 
car nous aimons dans notre nature être divertis et susciter des émotions, donner notre avis et voir 
nos amis passer un bon moment, ce qui parfois nous mènera à une dépression. (Voir "Le problème 
des réseaux sociaux") \\ Ces applications sont bien habillés pour pouvoir nous attirer et nous 
faire rester chez elles, l'une des principales et simple raison est l'amitié que l'on aura créé 
la dessus, en effet il s'agit d'un cercle vicieux car nous ne voulons pas perdre nos amis virtuels 
et surtout pas les intriguer et devoir s'expliquer face à nos amis que l'on voit dans la réalité de 
pourquoi nous quittons les réseaux sociaux, et eux de même. \\
Toutes ces applications qui nous font susciter beaucoup de réactions et qui font le nécessaire 
pour nous maintenir nous font nuire à notre liberté car elle nous font imaginer qu'elles peuvent 
prendre une grande place dans notre vie, et notre subconscient les écoutes. Cela devient une drogue
à cause de cette vilaine curiosité qui sera engendrée sur nous et nous ne pourrons plus nous y
séparer. \\
Comme autres exemples nous pouvons citer l'influence du regard des autres sur nous, nous privant de 
notre liberté de choix, car nous sommes implicitement forcés de suivre les codes de la société, de part
le comportement, l'expression, notre style vestimentaire, et notre avis. Nous, humains sommes créés pour
exprimer de l'amour envers chacun de nous, être sociable, et ces codes sont les premiers piliers de notre
société afin de faire des rencontres et nous entendre avec des individus. Il y a donc une liberté d'amour 
(et d'émotions) derrière beaucoup de liberté, tel que la liberté de choix, qui peuvent être privées.
\subsection*{III - Notre perception}
Malgré les deux exemples au dessus, expliquant pourquoi la liberté n'est pas présente dans notre
quotidien, je souhaite étendre mon idée de ce qu'est la liberté en disant que sa non-existence 
est lié à tout ce qu'il y a dans notre vie. \\ Notre manière de penser, la formalité de notre 
cerveau et structure de réseaux neuronal nous empêchent de penser librement, nous voyons le 
monde avec un aspect mathématique, parfois binaire, comme le fait de juger un individu comme 
mauvais ou bon, ce qui résume la vie de cette dernière, en oubliant une partie de tous ces actes
pour la mettre dans un casier prédéfinit avec d'autres personnes qui peuvent avoir commis des
actes plus dangereux. \\ En effet, malgré que le sujet de ce papier ne soit pas sur la vision 
mathématique du monde, j'aimerai pointer du doigt le problème de cette dernière, car la société 
l'utilise et cela à beaucoup de conséquences tel que la mise à l'écart des personnes ne rentrant
pas dans les cases proposées, car ce système essaye de prendre le plus d'individus en compte, mais 
ne pourra jamais tous les prendre. \\ Pour revenir au problème de la vision mathématique, cette 
vision nous empêche d'être ouvert d'esprit, et je pense que cette dernière ne nous mènera jamais
à une société complètement stable sur un long terme valant des millénaires, car, selon-moi, les 
mathématiques existent afin de représenter dans un langage universel et compréhensible par quiconque 
qui l'apprend, une idée partagée, pouvant être expliquée oralement. Elle simplifie la prise en compte 
et l'explication d'un phénomène ou d'un problème, mais la méthode appliquée par notre gouvernement 
et société n'arrivera pas à prendre tous les cas, comme le fait notre cerveau, en essayant de faire 
rentrer un 12 dans une liste pouvant accepter tous nombres de 0 à 11. \\
Cette vision mathématique de la vie donc, à travers notre cerveau nous empêche d'être libre 
mentalement, d'être libre de penser, cette vision que l'on nous apprend depuis notre plus 
jeune âge car la société l'aura souhaité, en différenciant une personne bonne et une autre
mauvaise, sur base de préjugés, nous empêche d'être ouvert d'esprit et de donc penser 
librement.
\footnote{ 
  NB: Je ne critique pas les mathématiques, mais les mathématiques utilisés dans notre société
  afin de trouver des solutions à des problèmes.
}

\section*{Conclusion}
La liberté, à proprement parlé, selon-moi n'existe pas et n'existera jamais, il s'agit d'une 
conception mentale nous aidant à nous sentir mieux, plus indépendant, pour faire valoir et augmenter 
notre propre estime. Un esclave peut être et se sentir encore plus libre qu'un maître. Je classerai
même la liberté comme une vision de vie, nous aidant à aller de l'avant, à vouloir entreprendre,
sans jugements ni conséquences. Il s'agit d'un mot ayant de très nombreuses significations, mais
il s'agirait concrètement de l'absence de contrainte mentale, car tout ce passe réellement dans 
notre vision et avis sur le monde. \\ Une philosophie intéressante qui pourrait être adoptée 
serait celle de la liberté intérieure, penser que l'on est libre, chacun, à l'intérieur de nous 
même. Elle nous aiderait à voir le monde plus positivement, faire un premier pas afin de quitter
nos addictions nous privant de liberté, vouloir imaginer, créer, innover, ne moins faire attention
au regard des autres.
\end{document}